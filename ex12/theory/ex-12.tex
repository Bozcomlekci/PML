\documentclass{tufte-handout}

%\geometry{showframe}% for debugging purposes -- displays the margins

\usepackage{amsmath}
\usepackage{amsthm}
\usepackage{amssymb}
\usepackage{xcolor}

\newtheorem{theorem}{Theorem}[section]
\newtheorem{corollary}{Corollary}[theorem]
\newtheorem{lemma}[theorem]{Lemma}
\newtheorem{notation}[theorem]{Notation}
\newtheorem{definition}[theorem]{Definition}
\newtheorem{question}[theorem]{Question}

\DeclareMathOperator*{\argmax}{arg\,max}
\DeclareMathOperator*{\argmin}{arg\,min}
% Set up the images/graphics package
\usepackage{graphicx}
\setkeys{Gin}{width=\linewidth,totalheight=\textheight,keepaspectratio}
\graphicspath{{graphics/}}

\title{Excercise-12}

\author{Batuhan, Oezcoemlekci (Matrikelnummer: 6300476) \\
        Aakarsh, Nair (Matrikelnummer: 6546577)}
\date{16 June 2023}  

% The following package makes prettier tables.  We're all about the bling!
\usepackage{booktabs}

% The units package provides nice, non-stacked fractions and better spacing
% for units.
\usepackage{units}

% The fancyvrb package lets us customize the formatting of verbatim
% environments.  We use a slightly smaller font.
\usepackage{fancyvrb}
\fvset{fontsize=\normalsize}

% Small sections of multiple columns
\usepackage{multicol}

% Provides paragraphs of dummy text
\usepackage{lipsum}

% These commands are used to pretty-print LaTeX commands
\newcommand{\doccmd}[1]{\texttt{\textbackslash#1}}% command name -- adds backslash automatically
\newcommand{\docopt}[1]{\ensuremath{\langle}\textrm{\textit{#1}}\ensuremath{\rangle}}% optional command argument
\newcommand{\docarg}[1]{\textrm{\textit{#1}}}% (required) command argument
\newenvironment{docspec}{\begin{quote}\noindent}{\end{quote}}% command specification environment
\newcommand{\docenv}[1]{\textsf{#1}}% environment name
\newcommand{\docpkg}[1]{\texttt{#1}}% package name
\newcommand{\doccls}[1]{\texttt{#1}}% document class name
\newcommand{\docclsopt}[1]{\texttt{#1}}% document class option name

\begin{document}

\maketitle% this prints the handout title, author, and date

%\printclassoptions
\section{Exercise}

\textbf{Theory Question} Parameter inference with exectation maximization.

Consider a linear Gaussian model with  the following structure:
\begin{itemize}
    \item $x_0 \sim \mathcal{N}(m_0(\theta), P_0(\theta))$
    \item $x_k \sim \mathcal{N}(A(\theta) x_{k-1}, Q(\theta))$
    \item $y_k \sim \mathcal{N}(H(\theta) x_{k}, R(\theta))$
\end{itemize}

\textbf{To-do:} Estimate $\theta$ for some data $y_{1:N}$ an alternate to 
maximizing the likelihood (discussed in lecture) is the EM-Algorithm.
It works as follows: 
\begin{enumerate}
    \item  Start from an initial guess $\theta^{(0)}$.
    \item For $n = 0, 1, 2, \dots$ do:
    \begin{enumerate}
        \item \textbf{E-Step:} Compute:  \\
        \begin{equation}
            \mathcal{Q}\left( \theta, \theta^{(n)} \right) := \int p(x_{0:N} | y_{0:N}, \theta^{(n)}) \log p(x_{0:N}, y_{0:N} | \theta) dx_{0:N}
        \end{equation}
            
        \item \textbf{M-Step:} Compute: $\theta^{(n+1)} := \argmax_{\theta} \mathcal{Q}(\theta, \theta^{(n)})$ 
    \end{enumerate}
\end{enumerate}

In a considered case of linear Gaussian state space model, $\mathcal{Q}(\theta, \theta^{(n)})$ can be computed analytically. 
It is of the form: 
    \begin{multline}
        \begin{aligned}
        \mathcal{Q}(^{\theta, \theta^{(n)}}) = \\ 
        & -\frac{1}{2} \log|2\pi P_0(\theta)| - \frac{N}{2} \log|2\pi Q(\theta)| - \frac{N}{2} \log|2\pi R(\theta)| \\
        & - \frac{1}{2} \operatorname{tr} \left( P_0^{-1}(\theta) \left[ P_0^{(s)} + (m_0^{(s)} - m_0(\theta))(m_0^{(s)} - m_0(\theta))^T \right] \right) \\ 
        & - \frac{1}{2} \operatorname{tr} \left( Q^{-1}(\theta) \left[  \Sigma -  CA(\theta)^T - A(\theta) C^T + A(\theta) \Phi A(\theta)^T \right] \right) \\ 
        & - \frac{1}{2} \operatorname{tr} \left( R^{-1}(\theta) \left[  D-  BH(\theta)^T - H(\theta) B^T + H(\theta) \Sigma H(\theta)^T \right] \right) \\ 
        \end{aligned}
    \end{multline}
Where the following quantities are computed form the results of the Rauch-Tung-Striebel smoother run with parameter values $\theta^{(n)}$:

\begin{multline*}
    \begin{aligned}
        \Sigma & = \frac{1}{N} \sum_{k=1}^N \left( P_k^{(s)} + m_k^{s} (m_{k}^{s})^T \right)   \\
        \Phi & = \frac{1}{N} \sum_{k=1}^N \left(P_{k-1}^{(s)} + m_{k-1}^{s} (m_{k-1}^{s})^T \right) \\
        B & = \frac{1}{N} \sum_{k=1}^N \left( y_k (m_{k}^{(s)})^T \right)\\
        C &= \frac{1}{N} \sum_{k=1}^{N} \left( P_k^{(s)} G_{k-1}^T + m_k^{(s)} (m_{k-1}^{s})^T \right) \\
        D &= \frac{1}{N} \sum_{k=1}^{N} \left( y_k y_{k}^T \right) 
    \end{aligned}
\end{multline*}

\emph{Exercise:} Let $\theta = (A, Q)$, that is the model parameters are exactly the full 
transition model matrices. Derive closed form updates for both $A$ and $Q$ by computing the M-step.


\begin{enumerate}[(a)]
    \item Compute $A$:
    We use the fact that trace is a linear operator.   
    \begin{multline}
        \begin{aligned}
            \frac{\partial Q}{\partial A} &= 0 \\
             & \implies \frac{\partial}{\partial A} \left[ -\frac{N}{2} \operatorname{tr} \left( Q^{-1} (\Sigma - CA^{T} - AC^T + A \Phi A^T)  \right) \right]  = 0 \\
             &\implies \frac{N}{2} \frac{\partial}{\partial A} \operatorname{tr}\left[ -Q^{-1}AC^T   - Q^{-1}CA^T  + Q^{-1}A \Phi A^T   \right] = 0 \\ 
             &\implies \frac{\partial}{\partial A} \operatorname{tr}\left[ -Q^{-1}AC^T   - Q^{-1}CA^T + Q^{-1}A \Phi A^T   \right] = 0 \\ 
             &\implies \frac{\partial}{\partial A} \operatorname{tr}\left[ Q^{-1}A \Phi A^T   \right] = \frac{\partial}{\partial A} \operatorname{tr}\left[ Q^{-1}AC^T  + Q^{-1}CA^T  \right]  
        \end{aligned}
    \end{multline} 

    Now consider the LHS: 
    \begin{multline}
       \begin{aligned} 
            \frac{\partial}{\partial A} \operatorname{tr}\left[ Q^{-1}A \Phi A^T   \right]  
       \end{aligned}
    \end{multline}
    This is known matrix differential form of the trace.
    \begin{equation*}
       \frac{\partial}{\partial{X_d}} \operatorname{tr}\left[ A_dX_dB_dX_d^TC_d \right] = A_d^{T}C_d^{T}X_dB_d^{T} + C_dA_dX_dB_d 
    \end{equation*}
    Where $A_d = Q^{-1}$, $B_d = \Phi$, $C_d = I$ and $X_d = A$.

    Thus we have  we use the symmetry of the postive semi-definte matrix inverse   $Q^{-1} = (Q^{-1})^T$ to simplify the LHS:
    \begin{multline}
       \begin{aligned} 
            \frac{\partial}{\partial A} \operatorname{tr}\left[ Q^{-1}A \Phi A^T   \right]   &= Q^{-T} I^T A \Phi^T + I Q^{-1} A \Phi \\
            &= Q^{-T} A \Phi^T + Q^{-1} A \Phi \\
            &= Q^{-T} A \Phi^T + Q^{-1} A \Phi \\
            &= 2 Q^{-1} A \Phi \\
       \end{aligned}
    \end{multline}

    For the RHS we can we can simplify the first term using:  

    \begin{multline}
    \begin{aligned}     
        \frac{\partial}{\partial X_d} \operatorname{tr}\left[ A_d X_d B_d \right] &=  A_d^T B_d^T 
    \end{aligned}
    \end{multline}

    With $A_d = (Q^{-1})$ and $X_d = A$ and $B_d = C^T$ for the first term we get:
    
    \begin{multline}
        \begin{aligned}
            \frac{\partial}{\partial A} \operatorname{tr}\left[ Q^{-1}AC^T  \right] &=  Q^{-T}C &= Q^{-1}C \\
        \end{aligned} 
    \end{multline}
    
    And for the second term we can use the following matrix identity we get: 
    \begin{multline}
    \begin{aligned}     
        \frac{\partial}{\partial X_d} \operatorname{tr}\left[ A_d X_d^T \right] &= A_d
    \end{aligned}
    \end{multline}
    With $A_d = (Q^{-1})C$ and $X_d = A$ the second term simplifies to:  
    \begin{multline}
        \begin{aligned}
            \frac{\partial}{\partial A} \operatorname{tr}\left[ Q^{-1}CA^T  \right] &=  Q^{-1}C \\
        \end{aligned} 
    \end{multline}
   
    Thus taken together , using the linearity of the trace operator  we can separate the terms we get: 

    \begin{multline}
        \begin{aligned}
            \frac{\partial}{\partial A} \operatorname{tr}\left[ Q^{-1}AC^T  + Q^{-1}CA^T  \right] &= Q^{-1}C + Q^{-1}C \\
            &= 2 Q^{-1}C \\    
        \end{aligned}
    \end{multline}

    Thus LHS and RHS together becomes: 
    \begin{multline}
        \begin{aligned}
            2 Q^{-1} A \Phi &= 2 Q^{-1} C \\
            Q^{-1} A \Phi &=  Q^{-1} C \\
            A \Phi &=  C \\
            A &=  C \Phi^{-1} 
        \end{aligned}
    \end{multline}        

    \item Compute $Q$:

    We again compute the critical point by computing the matrix derivative with respect to $Q$ and setting it to zero. Canceling the the constant
    from the linear operators for trace and derivative.
    
    Let $$Z =  \left[  \Sigma -  CA^{T} - A C^T + A \Phi A^T \right]$$

    \begin{multline}
        \begin{aligned}
           \frac{\partial} {\partial Q} \left[ -\frac{N}{2} \log |2\pi Q|  - \frac{N}{2} \operatorname{tr} \left( Q^{-1} Z \right)\right] = 0 \\
           \frac{\partial} {\partial Q} \left[ -\log |2\pi Q|  - \frac{\partial} {\partial Q} \operatorname{tr} \left( Q^{-1}  Z\right)\right] = 0 \\
           \frac{\partial} {\partial Q} \left[ -\log |2\pi Q| \right]  - \left[ \frac{\partial} {\partial Q} \operatorname{tr} \left( Q^{-1} Z \right) \right] = 0 \\
           \frac{\partial} {\partial Q} \left[ -\log |Q| \right]  - \left[ \frac{\partial} {\partial Q} \operatorname{tr} \left( Q^{-1} Z \right) \right] = 0 \\
           -\left[ \frac{\partial} {\partial Q} \operatorname{tr} \left( Q^{-1} Z \right) \right] = \frac{\partial} {\partial Q} \left[ \log |Q| \right]   \\ 
        \end{aligned}
    \end{multline}
   
    Consider the LHS, we use the following matrix differential form of the trace.
    \begin{multline}
        \begin{aligned}
            \frac{\partial}{\partial X_d} \operatorname{tr}(A_{d}X_{d}^{-1}B_{d}) &= -X_d^{-T}A_d^{T} B_d^T X_d^{-T}
        \end{aligned}
    \end{multline} 

    With $A_d = I$, $B_d = Z$ and $X_d = Q$ , and the fact that $Q^{-T} = Q^{-1}$ being inverse of PSD metrix we get:
   \begin{multline} 
    \begin{aligned}
        \frac{\partial}{\partial Q} (Q^{-1}Z) &= -Q^{-T}Z^T Q^{-T}  = - Q^{-1} Z^T Q^{-1}\\
    \end{aligned}
   \end{multline}

   We also note that $Z = Z^T$  as we have that $\Sigma$ and $\Phi$ are symmetric matrices.
   \begin{multline}
    \begin{aligned}
        Z^T &= (\Sigma - CA^T - AC^T + A \Phi A^T )^T &=  (\Sigma^T  - AC^T - C A^T  + A \Phi^T A^T)\\
        & = (\Sigma - CA^T - AC^T + A \Phi A^T ) &= Z 
    \end{aligned} 
 \end{multline}

Thus we get the LHS to be :
    \begin{multline}
        \begin{aligned}
            \frac{\partial}{\partial X_d} \operatorname{tr}(Q^{-1}Z) &= -Q^{-1} Z Q^{-1} 
        \end{aligned}
    \end{multline} 

For the RHS we use the matrix identity:
\begin{multline}
    \begin{aligned}
        \frac{\partial}{\partial X_d} \log |X_d| &= X_d^{-T}
    \end{aligned}
\end{multline}

With $X_d = Q$ we get the RHS to be:
\begin{multline}
    \begin{aligned}
        \frac{\partial}{\partial Q} \log |Q| &= Q^{-T} = Q^{-1}
    \end{aligned}
\end{multline}

Taken together we get 
\begin{multline}
    \begin{aligned}
    -Q^{-1} Z Q^{-1} &= -Q^{-1} \\
    \implies Q^{-1} Z Q^{-1} &= Q^{-1}  \\
    \implies Z Q^{-1} &= I 
    \end{aligned}
\end{multline}


Thus from the uniqueness of the inverse we get  
\begin{equation}
    Q = Z = (\Sigma - CA^T - AC^T + A \Phi A^T)
\end{equation}

\end{enumerate}

\bibliography{ex-12}
\bibliographystyle{plainnat}
\end{document}