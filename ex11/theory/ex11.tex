\documentclass{tufte-handout}

%\geometry{showframe}% for debugging purposes -- displays the margins

\usepackage{amsmath}
\usepackage{amsthm}
\usepackage{amssymb}
\usepackage{xcolor}

\newtheorem{theorem}{Theorem}[section]
\newtheorem{corollary}{Corollary}[theorem]
\newtheorem{lemma}[theorem]{Lemma}
\newtheorem{notation}[theorem]{Notation}
\newtheorem{definition}[theorem]{Definition}
\newtheorem{question}[theorem]{Question}

\DeclareMathOperator*{\argmax}{arg\,max}
\DeclareMathOperator*{\argmin}{arg\,min}
% Set up the images/graphics package
\usepackage{graphicx}
\setkeys{Gin}{width=\linewidth,totalheight=\textheight,keepaspectratio}
\graphicspath{{graphics/}}

\title{Excercise-11}
\author[ Batuhan, Oezcoemlekci(Matrikelnummer: 6300476) and Aakarsh, Nair( Matrikelnummer: 6546577)]{ML-4202: Probabilistic Machine Learning (Summer 2023)}
\date{16 June 2023}  

% The following package makes prettier tables.  We're all about the bling!
\usepackage{booktabs}

% The units package provides nice, non-stacked fractions and better spacing
% for units.
\usepackage{units}

% The fancyvrb package lets us customize the formatting of verbatim
% environments.  We use a slightly smaller font.
\usepackage{fancyvrb}
\fvset{fontsize=\normalsize}

% Small sections of multiple columns
\usepackage{multicol}

% Provides paragraphs of dummy text
\usepackage{lipsum}

% These commands are used to pretty-print LaTeX commands
\newcommand{\doccmd}[1]{\texttt{\textbackslash#1}}% command name -- adds backslash automatically
\newcommand{\docopt}[1]{\ensuremath{\langle}\textrm{\textit{#1}}\ensuremath{\rangle}}% optional command argument
\newcommand{\docarg}[1]{\textrm{\textit{#1}}}% (required) command argument
\newenvironment{docspec}{\begin{quote}\noindent}{\end{quote}}% command specification environment
\newcommand{\docenv}[1]{\textsf{#1}}% environment name
\newcommand{\docpkg}[1]{\texttt{#1}}% package name
\newcommand{\doccls}[1]{\texttt{#1}}% document class name
\newcommand{\docclsopt}[1]{\texttt{#1}}% document class option name

\begin{document}

\maketitle% this prints the handout title, author, and date

%\printclassoptions
\section{Exercise}

\begin{enumerate}[(a)]
    \item 
    
    \item Now we note that the function for $g$  for the filter covariance does not depend on the data. In fact, the Kalman filter covariance taks on the form of the so-called \emph{discrete-time algabraic Riccati equation(DARE)},. i.e an implicit equation of the form: 
    \begin{equation*}
        P_t = C^TP_{t-1}C - (C^T P_{t-1}U)(Z + U^TP_{t-1}U)^{-1}(U^TP_{t-1}C) + N
    \end{equation*}

    Derive the exact form of the Riccati equation by inserting the 
    predicted covariance matrix into the updated covariance matrix (similar to what you did with the mean). Identify the relationship
    between $(C, U, Z, N)$ and $(A, Q, H, R)$.

    
    \textbf{Hints:}
    \begin{itemize}
        \item Make sure that there is no $m^{-}$  or $P^{-}$ hidden in some auxilliary variables.
        \item You an write the update step for the covariance matrix $P_t = (I  - KH)P_t^{-}$ where $K$ is the Kalman gain and $I$ is the  idntity matrix. This should simplify things.
    \end{itemize}

    \textbf{Answer:}

   We start with the equations we know that is:
   \begin{equation}
        P_t^{-} = AP_{t-1}A^T + Q
        \label{predict-step}
   \end{equation}
   as well as :
   
   \begin{equation}
      P_t  = (I  - KH)P_t^{-}
      \label{update-step}
   \end{equation}

   We know the kalman gain is given by: 
   
   \begin{equation}
       K = P_t^{-}H^{t}(HP^{-}_t H^t + R)^{-1}
   \end{equation}
   
   We simplify the equation \ref{update-step} by first muliplying through then using the Woodburry identity which is given as:
   
   \begin{equation*}
    (A_w + U_wC_wV_w)^{-1} = A_w^{-1} - A_w^{-1}U_w(C_w^{-1} + V_wA_w^{-1}U_w)^{-1} V_wA_w^{-1}
   \end{equation*}

   Thus we get:
   \begin{equation}
        P_t = P_t^{-} -  P_t^{-}H^T(H P^{-}_{t}H^T + R)^{-1}H P_t^{-}
        \label{update-1}
   \end{equation}
   
   We note that the left hand side follows the form of the woodburry identity where $A_w = (P_t^{-})^{-1}$, $U_w = H^T$, $V_w = H$, $C = R^{-1}$.
   
   Thus we can simplify the LHS of \ref{update-1} to 

   \begin{equation}
       P_t = ((P_t^{-})^{-1} +   H^TR^{-1}H)^{-1}
       \label{simplified-update-1}
   \end{equation}

    Now substituting \ref{predict-step}  into \ref{simplified-update-1} we get:
    
    \begin{equation}
       P_t = ((Q + AP_{t-1}A^T)^{-1} + H^TR^{-1}H)^{-1}
    \end{equation}

    Using woodburry identity with $A_w = Q$ $U_w = A$, $C_w = P_{t-1}$, $V = A^T$ simplifying the inner inverse we get:
    
    \begin{equation}
       P_t = \left( (Q^{-1} - Q^{-1}A(P_{t-1}^{-1} + A^TQ^{-1}A)^{-1}A^{T}Q^{-1})+ H^TR^{-1}H \right)^{-1}
    \end{equation}
We can re arrange the terms collecting terms without $P_{t-1}$  to get: 

    \begin{equation}
       P_t = \left( (Q^{-1}  + H^TR^{-1}H)   - Q^{-1}A(P_{t-1}^{-1} + A^TQ^{-1}A)^{-1}A^{T}Q^{-1} \right)^{-1}
    \end{equation}

   Let $K_1 =(Q^{-1}  + H^TR^{-1}H)$ , then we note that we have, 
   $A_w = K_1$, $U_w = Q^{-1}A$, $C_w =-(P_{t-1}^{-1} + A^TQ^{-1}A)^{-1} $ and $V_w = A^TQ^{-1}$, Thus we can open the inverse to get:

  \begin{equation} 
    P_t = K_1^{-1}  - K_1^{-1} (Q^{-1}A) \left( -(P_{t-1}^{-1} + A^TQ^{-1}A) + A^TQ^{-1} K_1^{-1} Q^{-1}A \right)^{-1} (A^T Q^{-1}) K_1^{-1}
    \label{before-k2-expanded}
  \end{equation}

 Consider the inner inverse term:  
 \begin{equation}
     K_2 :=  (-(P_{t-1}^{-1} + A^TQ^{-1}A) + A^TQ^{-1} K_1^{-1} Q^{-1}A)^{-1} 
 \end{equation} 
  Simplifying it we get: 
  \begin{equation}
     K_2 =  -\left(P_{t-1}^{-1} + A^T(  Q^{-1} - Q^{-1} K_1^{-1} Q^{-1})A \right)^{-1} 
  \end{equation}
  Where $A_w = P_{t-1}^{-1}$, $U_w = A^T$, $C_w = (  Q^{-1} - Q^{-1} K_1^{-1} Q^{-1})$  and $V_w = A$, using woodburry identity we get: 

  \begin{equation} 
    K_2 = - \left( P_{t-1}  - P_{t-1}A^{T}( (Q^{-1} - Q^{-1}K_1 Q^{-1})^{-1} + A P_{t-1} A^T)^{-1}A P_{t-1}  \right)
  \end{equation}

  \begin{equation}
     K_2 =  -P_{t-1}  + P_{t-1}A^{T}( (Q^{-1} - Q^{-1}K_1 Q^{-1})^{-1} + A P_{t-1} A^T)^{-1}A P_{t-1}   
     \label{k2-expanded}
  \end{equation}

Substituting \ref{k2-expanded} into \ref{before-k2-expanded}  we get:

\begin{multline}
    P_t = K_1^{-1} - K_1^{-1} (Q^{-1}A) \left(-P_{t-1}  + P_{t-1}A^{T}( (Q^{-1} - Q^{-1}K_1 Q^{-1})^{-1} + A P_{t-1} A^T)^{-1}A P_{t-1}  \right) (A^T Q^{-1}) K_1^{-1}
\end{multline}

Thus $P_t$ becomes:
\begin{multline}
     P_t = \\ 
     K_1^{-1} (Q^{-1}A) P_{t-1} (A^T Q^{-1}) K_1^{-1}\\
   - (K_1^{-1} (Q^{-1}A)  
    P_{t-1}A^{T})( (Q^{-1} - Q^{-1}K_1 Q^{-1})^{-1} + A P_{t-1} A^T)^{-1} (A P_{t-1} (A^T Q^{-1}) K_1^{-1}) \\ 
    + K_1^{-1} 
\end{multline}

Comparing with the equation 
    \begin{equation*}
        P_t = C^TP_{t-1}C - (C^T P_{t-1}U)(Z + U^TP_{t-1}U)^{-1}(U^TP_{t-1}C) + N
    \end{equation*}

    We see that $C = A^TQ^{-1}K_1^{-1} $ and  $C^T = K_1^{-1} Q^{-1}A$ , $N = K_1^{-1}$, $Z = (Q^{-1} - Q^{-1}K_1 Q^{-1})^{-1} $, $U^T =A $ and $U = A^T$

    Substituting the value of $K_1=(Q^{-1}  + H^TR^{-1}H) $

    We get: 
    \begin{itemize}
        \item  $C = A^TQ^{-1}(Q^{-1} + H^T R^{-1}H)^{-1}$
        \item  $C^T = (Q^{-1} + H^T R^{-1}H^{-1})^{-1}Q^{-1}A$
        \item $Z  = (Q^{-1} - Q^{-1}(Q^{-1}  + H^TR^{-1}H) Q^{-1})^{-1} $
        \item $U^T = A , U = A^T$
    \end{itemize}
\end{enumerate}

We still need to show that the terms $C$ and $C^T$ are 
transposes of each other: 

\bibliography{05.06.2023}
\bibliographystyle{plainnat}
\end{document}
